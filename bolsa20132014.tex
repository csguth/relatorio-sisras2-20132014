%% abtex2-modelo-projeto-pesquisa.tex, v-1.9.1 laurocesar
%% Copyright 2012-2013 by abnTeX2 group at http://abntex2.googlecode.com/ 
%%
%% This work may be distributed and/or modified under the
%% conditions of the LaTeX Project Public License, either version 1.3
%% of this license or (at your option) any later version.
%% The latest version of this license is in
%%   http://www.latex-project.org/lppl.txt
%% and version 1.3 or later is part of all distributions of LaTeX
%% version 2005/12/01 or later.
%%
%% This work has the LPPL maintenance status `maintained'.
%% 
%% The Current Maintainer of this work is the abnTeX2 team, led
%% by Lauro César Araujo. Further information are available on 
%% http://abntex2.googlecode.com/
%%
%% This work consists of the files abntex2-modelo-projeto-pesquisa.tex
%% and abntex2-modelo-references.bib
%%

% ------------------------------------------------------------------------
% ------------------------------------------------------------------------
% abnTeX2: Modelo de Projeto de pesquisa em conformidade com 
% ABNT NBR 15287:2011 Informação e documentação - Projeto de pesquisa -
% Apresentação 
% ------------------------------------------------------------------------ 
% ------------------------------------------------------------------------

\documentclass[
	% -- opções da classe memoir --
	12pt,				% tamanho da fonte
	openright,			% capítulos começam em pág ímpar (insere página vazia caso preciso)
	twoside,			% para impressão em verso e anverso. Oposto a oneside
	a4paper,			% tamanho do papel. 
	% -- opções da classe abntex2 --
	%chapter=TITLE,		% títulos de capítulos convertidos em letras maiúsculas
	%section=TITLE,		% títulos de seções convertidos em letras maiúsculas
	%subsection=TITLE,	% títulos de subseções convertidos em letras maiúsculas
	%subsubsection=TITLE,% títulos de subsubseções convertidos em letras maiúsculas
	% -- opções do pacote babel --
	english,			% idioma adicional para hifenização
	french,				% idioma adicional para hifenização
	spanish,			% idioma adicional para hifenização
	brazil,				% o último idioma é o principal do documento
	]{abntex2}

% ---
% PACOTES
% ---

% ---
% Pacotes fundamentais 
% ---
\usepackage{lmodern}			% Usa a fonte Latin Modern
\usepackage[T1]{fontenc}		% Selecao de codigos de fonte.
\usepackage[utf8]{inputenc}		% Codificacao do documento (conversão automática dos acentos)
\usepackage{indentfirst}		% Indenta o primeiro parágrafo de cada seção.
\usepackage{color}				% Controle das cores
\usepackage{graphicx}			% Inclusão de gráficos
\usepackage{microtype} 			% para melhorias de justificação
% ---

% ---
% Pacotes adicionais, usados apenas no âmbito do Modelo Canônico do abnteX2
% ---
\usepackage{lipsum}				% para geração de dummy text
% ---

% ---
% Pacotes de citações
% ---
\usepackage[brazilian,hyperpageref]{backref}	 % Paginas com as citações na bibl
\usepackage[alf]{abntex2cite}	% Citações padrão ABNT

% ---
% Customizacoes
% ---
\usepackage{sisras2}

% --- 
% CONFIGURAÇÕES DE PACOTES
% --- 

% ---
% Configurações do pacote backref
% Usado sem a opção hyperpageref de backref
\renewcommand{\backrefpagesname}{Citado na(s) página(s):~}
% Texto padrão antes do número das páginas
\renewcommand{\backref}{}
% Define os textos da citação
\renewcommand*{\backrefalt}[4]{
	\ifcase #1 %
		Nenhuma citação no texto.%
	\or
		Citado na página #2.%
	\else
		Citado #1 vezes nas páginas #2.%
	\fi}%
% ---

% ---
% Informações de dados para CAPA e FOLHA DE ROSTO
% ---
\titulo{Relatório de Atividades de Bolsista SisRas2\\(Junho de 2013 a Janeiro de 2014)}
\autor{Chrystian de Sousa Guth}
\local{Florianópolis}
\data{Janeiro de 2014}
\instituicao{Universidade Federal de Santa Catarina}
\tipotrabalho{Tese (Doutorado)}
% O preambulo deve conter o tipo do trabalho, o objetivo, 
% o nome da instituição e a área de concentração 
\preambulo{}
% ---

% ---
% Configurações de aparência do PDF final

% alterando o aspecto da cor azul
\definecolor{blue}{RGB}{41,5,195}

% informações do PDF
\makeatletter
\hypersetup{
     	%pagebackref=true,
		pdftitle={\@title}, 
		pdfauthor={\@author},
    	pdfsubject={\imprimirpreambulo},
	    pdfcreator={LaTeX with abnTeX2},
		pdfkeywords={abnt}{latex}{abntex}{abntex2}{projeto de pesquisa}, 
		colorlinks=true,       		% false: boxed links; true: colored links
    	linkcolor=blue,          	% color of internal links
    	citecolor=blue,        		% color of links to bibliography
    	filecolor=magenta,      		% color of file links
		urlcolor=blue,
		bookmarksdepth=4
}
\makeatother
% --- 

% --- 
% Espaçamentos entre linhas e parágrafos 
% --- 

% O tamanho do parágrafo é dado por:
\setlength{\parindent}{1.3cm}

% Controle do espaçamento entre um parágrafo e outro:
\setlength{\parskip}{0.2cm}  % tente também \onelineskip

% ---
% compila o indice
% ---
\makeindex
% ---

% ----
% Início do documento
% ----
\begin{document}

% Retira espaço extra obsoleto entre as frases.
\frenchspacing 

% ----------------------------------------------------------
% ELEMENTOS PRÉ-TEXTUAIS
% ----------------------------------------------------------
% \pretextual

% ---
% Capa
% ---
\imprimircapa
% ---

% ---
% Folha de rosto
% ---
%\imprimirfolhaderosto
% ---

% ---
% NOTA DA ABNT NBR 15287:2011, p. 4:
%  ``Se exigido pela entidade, apresentar os dados curriculares do autor em
%     folha ou página distinta após a folha de rosto.''
% ---

% ---
% inserir lista de ilustrações
% ---
%\pdfbookmark[0]{\listfigurename}{lof}
%\listoffigures*
%\cleardoublepage
% ---

% ---
% inserir lista de tabelas
% ---
%\pdfbookmark[0]{\listtablename}{lot}
%\listoftables*
%\cleardoublepage
% ---

% ---
% inserir lista de abreviaturas e siglas
% ---
%\begin{siglas}
%  \item[ABNT] Associação Brasileira de Normas Técnicas
%  \item[abnTeX] ABsurdas Normas para TeX
%\end{siglas}
% ---

% ---
% inserir lista de símbolos
% ---
%\begin{simbolos}
%  \item[$ \Gamma $] Letra grega Gama
%  \item[$ \Lambda $] Lambda
%  \item[$ \zeta $] Letra grega minúscula zeta
%  \item[$ \in $] Pertence
%\end{simbolos}
% ---

% ---
% inserir o sumario
% ---
%\pdfbookmark[0]{\contentsname}{toc}
%\tableofcontents*
%\cleardoublepage
% ---


% ----------------------------------------------------------
% ELEMENTOS TEXTUAIS
% ----------------------------------------------------------
\textual


\chapter{Relatório de Atividades do Bolsista}
% ----------------------------------------------------------
% Identificação
% ----------------------------------------------------------
\section{Identificação}

\begin{itemize}
	\item \textbf{Nome:} Chrystian de Sousa Guth;
	\item \textbf{Local de Trabalho:} Universidade Federal de Santa Catarina;
	\item \textbf{Título do Plano de Trabalho:} Avaliação do Impacto do Atraso das Interconexões na Análise de \textit{Timing} no Contexto de uma Ferramenta de \textit{Gate Sizing};
	\item \textbf{Tipo de Bolsa:} ITI-A;
	\item \textbf{Número do Processo da Bolsa:} 182980/2013-8;
	\item \textbf{Período:} Julho de 2013 - Janeiro de 2014;
	\item \textbf{Orientador:} José Luís Almada Güntzel (INE/UFSC);
	\item \textbf{Coordenador do Projeto:} Ricardo Augusto da Luz Reis (II/UFRGS).
\end{itemize}

% ----------------------------------------------------------
% Resumo 
% ----------------------------------------------------------
\section{Resumo}
Este documento relata as atividades realizadas pelo bolsista Chrystian de Sousa Guth no período de Junho de 2013 a Janeiro de 2014, no contexto de bolsa ITI-A associada ao ``Projeto SisRAS2 - Sistemas Computacionais com Capacidade de Confiabilidade, Disponibilidade e Utilidade (RAS) 2''.

Conforme previsto no plano de trabalho da bolsa, entre Junho de 2013 e Janeiro de 2013 o bolsista realizou um estudo para compreensão das técnicas de análise de \textit{timing} estática (\textit{STA: Static Timing Analysis}) aplicadas no contexto de uma ferramenta de otimização para fluxo industrial.

Nos primeiros meses (entre Junho e Setembro de 2013), o aluno desenvolveu uma ferramenta de STA utilizando a infraestrutura disponibilizada pela competição de \textit{sizing} discreto do ISPD 2013 \cite{Contest2013}, que levava em consideração o atraso das interconexões e a degradação do \textit{slew} através destas. Posteriormente, entre Outubro e Novembro de 2013, foram realizados experimentos a fim de validar a ferramenta, a qual foi documentada na forma de uma monografia de conclusão de curso entitulada ``Análise de \textit{Timing} Estática e a Avaliação do Impacto do Atraso das Interconexões em Circuitos Digitais'', com defesa realizada em Novembro de 2013. Finalmente, entre Dezembro de 2013 e Janeiro de 2014, a infraestrutura implementada até então foi incorporada à uma técnica de otimização para fluxo \textit{Standard Cell} conhecida como \textit{gate sizing}.

\begin{center}
\textit{\textbf{Palavras-chave:} automação de projeto eletrônico, biblioteca standard cell, análise de timing estática, complementary metal-oxide semiconductor, gate sizing.}
\end{center}

\section{Introdução}

\section{Materiais e Métodos}

\section{Revisão Bibliográfica}

\section{Experimentos Desenvolvidos, Resultados e Discussões}

% ---
% Finaliza a parte no bookmark do PDF
% para que se inicie o bookmark na raiz
% e adiciona espaço de parte no Sumário
% ---
\phantompart

% ---
% Conclusão
% ---
\chapter*[Considerações finais]{Considerações finais}
\addcontentsline{toc}{chapter}{Considerações finais}

\lipsum[31-33]

% ----------------------------------------------------------
% ELEMENTOS PÓS-TEXTUAIS
% ----------------------------------------------------------
\postextual

% ----------------------------------------------------------
% Referências bibliográficas
% ----------------------------------------------------------
\bibliography{bolsa20132014}


\end{document}
